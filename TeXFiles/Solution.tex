\documentclass[12pt, a4paper]{article}
\usepackage[margin=1in]{geometry}
\usepackage{mathtext}
\usepackage[T2A]{fontenc}
\usepackage[utf8]{inputenc}
\usepackage{amsmath, amssymb}
\usepackage[english, russian]{babel}

\title{Основы дифференцального исчисления (2 том)}
\author{Фазлеев Ян}

\begin{document}
\maketitle
\setlength{\parindent}{0pt}
\large

\hspace{1cm}Дорогие читатели, это моя первая серьёзная (честно) работа по математическому анализу (опустим тот факт, что это 2 том моей книги). Здесь я хотел бы обсудить с вами важнейший раздел математического анализа -- \textbf{Дифференцальное исчисление}. Я уверен, что эту книгу читают люди, прочувствовашие всю красоту матанализа и изучившие достаточное количество теорем, поэтому о производных элементарных функций я даже не буду говорить, ведь все они очевидны любому советскому детсадовцу, но, если вам вдруг что-то не очевидно, то примите мои соболезнования и обязательно изучите учебники Редкозубова, Зорича и Иванова.\vspace{0.5cm}

\hspace{1cm}В качестве несложного примера продифференцируем следующее выражение:

$$  (  \sin  (  {  (  \cos  (  {  {x}  }  ^  {  {  3}  }  )  )  }  \cdot {  (  \ln  (  \tan  (  \frac {  {  1}  }  {  {x}  }  )  )  )  }  )  )  $$
\hspace{1cm}Нетрудно убедиться в том, что:
$$ ( (  \frac {  {  1}  }  {  {x}  }  ) )_{x}' = $$
$$ =  (  \frac {  { -1}  }  {  (  {  {x}  }  ^  {  {  2}  }  )  }  )  $$

\hspace{1cm}Путём нетрудных вычислений получим:
$$ ( (  \tan  (  \frac {  {  1}  }  {  {x}  }  )  ) )_{x}' = $$
$$ =  (  {  (  \frac {  { -1}  }  {  (  {  {x}  }  ^  {  {  2}  }  )  }  )  }  \cdot {  (  \frac {  {  1}  }  {  (  {  (  \cos  (  \frac {  {  1}  }  {  {x}  }  )  )  }  ^  {  {  2}  }  )  }  )  }  )  $$

\hspace{1cm}Следующее утверждение нам выдаст ChatGpt:
$$ ( (  \ln  (  \tan  (  \frac {  {  1}  }  {  {x}  }  )  )  ) )_{x}' = $$
$$ =  (  {  (  {  (  \frac {  { -1}  }  {  (  {  {x}  }  ^  {  {  2}  }  )  }  )  }  \cdot {  (  \frac {  {  1}  }  {  (  {  (  \cos  (  \frac {  {  1}  }  {  {x}  }  )  )  }  ^  {  {  2}  }  )  }  )  }  )  }  \cdot {  (  \frac {  {  1}  }  {  (  \tan  (  \frac {  {  1}  }  {  {x}  }  )  )  }  )  }  )  $$

\hspace{1cm}По следствию из Китайской теоремы об остатках получаем:
$$ ( (  {  {x}  }  ^  {  {  3}  }  ) )_{x}' = $$
$$ =  (  {  {  3}  }  \cdot {  (  {  {x}  }  ^  {  {  2}  }  )  }  )  $$

\hspace{1cm}Как сказал бы А. А. Калиниченко -- "Вам это должны были в школе рассказать":
$$ ( (  \cos  (  {  {x}  }  ^  {  {  3}  }  )  ) )_{x}' = $$
$$ =  (  {  (  {  {  3}  }  \cdot {  (  {  {x}  }  ^  {  {  2}  }  )  }  )  }  \cdot {  (  {  {  0}  }  -  {  (  \sin  (  {  {x}  }  ^  {  {  3}  }  )  )  }  )  }  )  $$

\hspace{1cm}Следующий факт является ключом для понимания всего анализа в целом:
$$ ( (  {  (  \cos  (  {  {x}  }  ^  {  {  3}  }  )  )  }  \cdot {  (  \ln  (  \tan  (  \frac {  {  1}  }  {  {x}  }  )  )  )  }  ) )_{x}' = $$
$$ =  (  {  (  {  (  {  (  {  {  3}  }  \cdot {  (  {  {x}  }  ^  {  {  2}  }  )  }  )  }  \cdot {  (  {  {  0}  }  -  {  (  \sin  (  {  {x}  }  ^  {  {  3}  }  )  )  }  )  }  )  }  \cdot {  (  \ln  (  \tan  (  \frac {  {  1}  }  {  {x}  }  )  )  )  }  )  }  +  {  (  {  (  \cos  (  {  {x}  }  ^  {  {  3}  }  )  )  }  \cdot {  (  {  (  {  (  \frac {  { -1}  }  {  (  {  {x}  }  ^  {  {  2}  }  )  }  )  }  \cdot {  (  \frac {  {  1}  }  {  (  {  (  \cos  (  \frac {  {  1}  }  {  {x}  }  )  )  }  ^  {  {  2}  }  )  }  )  }  )  }  \cdot {  (  \frac {  {  1}  }  {  (  \tan  (  \frac {  {  1}  }  {  {x}  }  )  )  }  )  }  )  }  )  }  )  $$

\hspace{1cm}Нетрудно убедиться в том, что:
$$ ( (  \sin  (  {  (  \cos  (  {  {x}  }  ^  {  {  3}  }  )  )  }  \cdot {  (  \ln  (  \tan  (  \frac {  {  1}  }  {  {x}  }  )  )  )  }  )  ) )_{x}' = $$
$$ =  (  {  (  {  (  {  (  {  (  {  {  3}  }  \cdot {  (  {  {x}  }  ^  {  {  2}  }  )  }  )  }  \cdot {  (  {  {  0}  }  -  {  (  \sin  (  {  {x}  }  ^  {  {  3}  }  )  )  }  )  }  )  }  \cdot {  (  \ln  (  \tan  (  \frac {  {  1}  }  {  {x}  }  )  )  )  }  )  }  +  {  (  {  (  \cos  (  {  {x}  }  ^  {  {  3}  }  )  )  }  \cdot {  (  {  (  {  (  \frac {  { -1}  }  {  (  {  {x}  }  ^  {  {  2}  }  )  }  )  }  \cdot {  (  \frac {  {  1}  }  {  (  {  (  \cos  (  \frac {  {  1}  }  {  {x}  }  )  )  }  ^  {  {  2}  }  )  }  )  }  )  }  \cdot {  (  \frac {  {  1}  }  {  (  \tan  (  \frac {  {  1}  }  {  {x}  }  )  )  }  )  }  )  }  )  }  )  }  \cdot {  (  \cos  (  {  (  \cos  (  {  {x}  }  ^  {  {  3}  }  )  )  }  \cdot {  (  \ln  (  \tan  (  \frac {  {  1}  }  {  {x}  }  )  )  )  }  )  )  }  )  $$


\end{document}
